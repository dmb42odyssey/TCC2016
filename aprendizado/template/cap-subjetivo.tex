%% ------------------------------------------------------------------------- %%
\section{Aprendizado}
\subsection{Desafios e Frustra��es}
\label{desfru}

\par O maior desafio que encontrei nesse trabalho foi na obten��o de dados. N�o imaginava que existisse, para a minha felicidade, uma ferramenta que auxiliasse na navega��o automatizada de websites. S�o poucas as disciplinas no BCC onde a obten��o de dados tamb�m � um problema a ser resolvido. Considero tamb�m, que aprender os paradigmas e a utiliza��o de bibliotecas ao programar para o ambiente nativo Android em alguns meses, foi um tarefa bem �rdua e arriscada, ainda mais pela completa falta de experi�ncia, mas recompensadora no final.
\par De modo geral, estou bem satisfeito com o resultado do trabalho, e a minha �nica frustra��o � a falta de tempo para implementar m�todos mais robustos: Seja na categoriza��o dos dados ou no c�lculo de outros valores. A falta de tempo se deve n�o s� ao desafio citado no primeiro par�grafo, como tamb�m no design da interface do usu�rio, pois � algo que um aluno do BCC n�o est� acostumado a fazer. � muito dif�cil, apenas uma pessoa, cobrir todos os aspectos (Performance, Design, Implementa��o, etc..) de um aplicativo com excel�ncia. Entretanto, sinto que foi dado os primeiros passos para algo maior no futuro.

\subsection{Disciplinas relevantes}
\label{rel}

\begin{itemize}  
\item \textbf{MAC0209} - Modelagem e Simula��o 
\par Muito dos EPs eram a respeito de tratamento de dados a partir de um sensor. Essa disciplina me ajudou no filtro dos dados que s�o obtidos nesse trabalho.
\item \textbf{MAC0121} - Algoritmos e Estruturas de Dados I (Grade Antiga)
\par Essencial para qualquer aluno do BCC.
\item \textbf{MAC0242} - Laborat�rio de Programa��o II (Grade Antiga)
\par Foi essencial pro meu aprendizado da linguagem Java e no paradigma de programa��o orientada a objetos.
\end{itemize}


